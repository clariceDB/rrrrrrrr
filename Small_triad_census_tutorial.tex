\documentclass[]{article}
\usepackage{lmodern}
\usepackage{amssymb,amsmath}
\usepackage{ifxetex,ifluatex}
\usepackage{fixltx2e} % provides \textsubscript
\ifnum 0\ifxetex 1\fi\ifluatex 1\fi=0 % if pdftex
  \usepackage[T1]{fontenc}
  \usepackage[utf8]{inputenc}
\else % if luatex or xelatex
  \ifxetex
    \usepackage{mathspec}
  \else
    \usepackage{fontspec}
  \fi
  \defaultfontfeatures{Ligatures=TeX,Scale=MatchLowercase}
\fi
% use upquote if available, for straight quotes in verbatim environments
\IfFileExists{upquote.sty}{\usepackage{upquote}}{}
% use microtype if available
\IfFileExists{microtype.sty}{%
\usepackage{microtype}
\UseMicrotypeSet[protrusion]{basicmath} % disable protrusion for tt fonts
}{}
\usepackage[margin=1in]{geometry}
\usepackage{hyperref}
\hypersetup{unicode=true,
            pdftitle={R Notebook},
            pdfborder={0 0 0},
            breaklinks=true}
\urlstyle{same}  % don't use monospace font for urls
\usepackage{color}
\usepackage{fancyvrb}
\newcommand{\VerbBar}{|}
\newcommand{\VERB}{\Verb[commandchars=\\\{\}]}
\DefineVerbatimEnvironment{Highlighting}{Verbatim}{commandchars=\\\{\}}
% Add ',fontsize=\small' for more characters per line
\usepackage{framed}
\definecolor{shadecolor}{RGB}{248,248,248}
\newenvironment{Shaded}{\begin{snugshade}}{\end{snugshade}}
\newcommand{\AlertTok}[1]{\textcolor[rgb]{0.94,0.16,0.16}{#1}}
\newcommand{\AnnotationTok}[1]{\textcolor[rgb]{0.56,0.35,0.01}{\textbf{\textit{#1}}}}
\newcommand{\AttributeTok}[1]{\textcolor[rgb]{0.77,0.63,0.00}{#1}}
\newcommand{\BaseNTok}[1]{\textcolor[rgb]{0.00,0.00,0.81}{#1}}
\newcommand{\BuiltInTok}[1]{#1}
\newcommand{\CharTok}[1]{\textcolor[rgb]{0.31,0.60,0.02}{#1}}
\newcommand{\CommentTok}[1]{\textcolor[rgb]{0.56,0.35,0.01}{\textit{#1}}}
\newcommand{\CommentVarTok}[1]{\textcolor[rgb]{0.56,0.35,0.01}{\textbf{\textit{#1}}}}
\newcommand{\ConstantTok}[1]{\textcolor[rgb]{0.00,0.00,0.00}{#1}}
\newcommand{\ControlFlowTok}[1]{\textcolor[rgb]{0.13,0.29,0.53}{\textbf{#1}}}
\newcommand{\DataTypeTok}[1]{\textcolor[rgb]{0.13,0.29,0.53}{#1}}
\newcommand{\DecValTok}[1]{\textcolor[rgb]{0.00,0.00,0.81}{#1}}
\newcommand{\DocumentationTok}[1]{\textcolor[rgb]{0.56,0.35,0.01}{\textbf{\textit{#1}}}}
\newcommand{\ErrorTok}[1]{\textcolor[rgb]{0.64,0.00,0.00}{\textbf{#1}}}
\newcommand{\ExtensionTok}[1]{#1}
\newcommand{\FloatTok}[1]{\textcolor[rgb]{0.00,0.00,0.81}{#1}}
\newcommand{\FunctionTok}[1]{\textcolor[rgb]{0.00,0.00,0.00}{#1}}
\newcommand{\ImportTok}[1]{#1}
\newcommand{\InformationTok}[1]{\textcolor[rgb]{0.56,0.35,0.01}{\textbf{\textit{#1}}}}
\newcommand{\KeywordTok}[1]{\textcolor[rgb]{0.13,0.29,0.53}{\textbf{#1}}}
\newcommand{\NormalTok}[1]{#1}
\newcommand{\OperatorTok}[1]{\textcolor[rgb]{0.81,0.36,0.00}{\textbf{#1}}}
\newcommand{\OtherTok}[1]{\textcolor[rgb]{0.56,0.35,0.01}{#1}}
\newcommand{\PreprocessorTok}[1]{\textcolor[rgb]{0.56,0.35,0.01}{\textit{#1}}}
\newcommand{\RegionMarkerTok}[1]{#1}
\newcommand{\SpecialCharTok}[1]{\textcolor[rgb]{0.00,0.00,0.00}{#1}}
\newcommand{\SpecialStringTok}[1]{\textcolor[rgb]{0.31,0.60,0.02}{#1}}
\newcommand{\StringTok}[1]{\textcolor[rgb]{0.31,0.60,0.02}{#1}}
\newcommand{\VariableTok}[1]{\textcolor[rgb]{0.00,0.00,0.00}{#1}}
\newcommand{\VerbatimStringTok}[1]{\textcolor[rgb]{0.31,0.60,0.02}{#1}}
\newcommand{\WarningTok}[1]{\textcolor[rgb]{0.56,0.35,0.01}{\textbf{\textit{#1}}}}
\usepackage{graphicx,grffile}
\makeatletter
\def\maxwidth{\ifdim\Gin@nat@width>\linewidth\linewidth\else\Gin@nat@width\fi}
\def\maxheight{\ifdim\Gin@nat@height>\textheight\textheight\else\Gin@nat@height\fi}
\makeatother
% Scale images if necessary, so that they will not overflow the page
% margins by default, and it is still possible to overwrite the defaults
% using explicit options in \includegraphics[width, height, ...]{}
\setkeys{Gin}{width=\maxwidth,height=\maxheight,keepaspectratio}
\IfFileExists{parskip.sty}{%
\usepackage{parskip}
}{% else
\setlength{\parindent}{0pt}
\setlength{\parskip}{6pt plus 2pt minus 1pt}
}
\setlength{\emergencystretch}{3em}  % prevent overfull lines
\providecommand{\tightlist}{%
  \setlength{\itemsep}{0pt}\setlength{\parskip}{0pt}}
\setcounter{secnumdepth}{0}
% Redefines (sub)paragraphs to behave more like sections
\ifx\paragraph\undefined\else
\let\oldparagraph\paragraph
\renewcommand{\paragraph}[1]{\oldparagraph{#1}\mbox{}}
\fi
\ifx\subparagraph\undefined\else
\let\oldsubparagraph\subparagraph
\renewcommand{\subparagraph}[1]{\oldsubparagraph{#1}\mbox{}}
\fi

%%% Use protect on footnotes to avoid problems with footnotes in titles
\let\rmarkdownfootnote\footnote%
\def\footnote{\protect\rmarkdownfootnote}

%%% Change title format to be more compact
\usepackage{titling}

% Create subtitle command for use in maketitle
\providecommand{\subtitle}[1]{
  \posttitle{
    \begin{center}\large#1\end{center}
    }
}

\setlength{\droptitle}{-2em}

  \title{R Notebook}
    \pretitle{\vspace{\droptitle}\centering\huge}
  \posttitle{\par}
    \author{}
    \preauthor{}\postauthor{}
    \date{}
    \predate{}\postdate{}
  

\begin{document}
\maketitle

\begin{Shaded}
\begin{Highlighting}[]
\NormalTok{pacman}\OperatorTok{::}\KeywordTok{p_load}\NormalTok{(randomNames, tidygraph, tidyverse, igraph, ggraph, ggplot)}
\end{Highlighting}
\end{Shaded}

\begin{verbatim}
## Warning: package 'ggplot' is not available (for R version 3.5.1)
\end{verbatim}

\begin{verbatim}
## Warning in p_install(package, character.only = TRUE, ...):
\end{verbatim}

\begin{verbatim}
## Warning in library(package, lib.loc = lib.loc, character.only = TRUE,
## logical.return = TRUE, : there is no package called 'ggplot'
\end{verbatim}

\begin{verbatim}
## Warning in pacman::p_load(randomNames, tidygraph, tidyverse, igraph, ggraph, : Failed to install/load:
## ggplot
\end{verbatim}

\begin{Shaded}
\begin{Highlighting}[]
\NormalTok{triad_names<-}\StringTok{ }\KeywordTok{c}\NormalTok{(}\StringTok{"003 "}\NormalTok{,}\StringTok{"012 "}\NormalTok{,}\StringTok{"102 "}\NormalTok{,}\StringTok{"021D"}\NormalTok{,}\StringTok{"021U"}\NormalTok{,}\StringTok{"021C"}\NormalTok{,}\StringTok{"111D"}\NormalTok{,}\StringTok{"111U"}\NormalTok{,}\StringTok{"030T"}\NormalTok{,}\StringTok{"030C"}\NormalTok{,}\StringTok{"201 "}\NormalTok{,}\StringTok{"120D"}\NormalTok{,}\StringTok{"120U"}\NormalTok{,}\StringTok{"120C"}\NormalTok{,}\StringTok{"210 "}\NormalTok{,}\StringTok{"300 "}\NormalTok{)}
\end{Highlighting}
\end{Shaded}

\#\#Using Triad Census

This function counts for each triple of vertices, the number of times
each of the 16 possible states occur.

\begin{itemize}
\tightlist
\item
  003

  \begin{itemize}
  \tightlist
  \item
    A,B,C, the empty graph.
  \end{itemize}
\item
  012

  \begin{itemize}
  \tightlist
  \item
    A-\textgreater{}B, C, the graph with a single directed edge.
  \end{itemize}
\item
  102

  \begin{itemize}
  \tightlist
  \item
    A\textless{}-\textgreater{}B, C, the graph with a mutual connection
    between two vertices.
  \end{itemize}
\item
  021D

  \begin{itemize}
  \tightlist
  \item
    A\textless{}-B-\textgreater{}C, the out-star.
  \end{itemize}
\item
  021U

  \begin{itemize}
  \tightlist
  \item
    A-\textgreater{}B\textless{}-C, the in-star.
  \end{itemize}
\item
  021C

  \begin{itemize}
  \tightlist
  \item
    A-\textgreater{}B-\textgreater{}C, directed line.
  \end{itemize}
\item
  111D

  \begin{itemize}
  \tightlist
  \item
    A\textless{}-\textgreater{}B\textless{}-C.
  \end{itemize}
\item
  111U

  \begin{itemize}
  \tightlist
  \item
    A\textless{}-\textgreater{}B-\textgreater{}C.
  \end{itemize}
\item
  030T

  \begin{itemize}
  \tightlist
  \item
    A-\textgreater{}B\textless{}-C, A-\textgreater{}C.
  \end{itemize}
\item
  030C

  \begin{itemize}
  \tightlist
  \item
    A\textless{}-B\textless{}-C, A-\textgreater{}C.
  \end{itemize}
\item
  201

  \begin{itemize}
  \tightlist
  \item
    A\textless{}-\textgreater{}B\textless{}-\textgreater{}C.
  \end{itemize}
\item
  120D

  \begin{itemize}
  \tightlist
  \item
    A\textless{}-B-\textgreater{}C, A\textless{}-\textgreater{}C.
  \end{itemize}
\item
  120U

  \begin{itemize}
  \tightlist
  \item
    A-\textgreater{}B\textless{}-C, A\textless{}-\textgreater{}C.
  \end{itemize}
\item
  120C

  \begin{itemize}
  \tightlist
  \item
    A-\textgreater{}B-\textgreater{}C, A\textless{}-\textgreater{}C.
  \end{itemize}
\item
  210

  \begin{itemize}
  \tightlist
  \item
    A-\textgreater{}B\textless{}-\textgreater{}C,
    A\textless{}-\textgreater{}C.
  \end{itemize}
\item
  300

  \begin{itemize}
  \tightlist
  \item
    A\textless{}-\textgreater{}B\textless{}-\textgreater{}C,
    A\textless{}-\textgreater{}C, the complete graph.
  \end{itemize}
\end{itemize}

\#\#\#Smaller graphs

\begin{Shaded}
\begin{Highlighting}[]
\NormalTok{one <-}\StringTok{ }\KeywordTok{graph_from_literal}\NormalTok{(A}\OperatorTok{-+}\NormalTok{B, C}\OperatorTok{--}\NormalTok{B)}

\KeywordTok{data.frame}\NormalTok{(}\DataTypeTok{tri =} \KeywordTok{triad_census}\NormalTok{(one), }\DataTypeTok{name =}\NormalTok{ triad_names)}
\end{Highlighting}
\end{Shaded}

\begin{verbatim}
##    tri name
## 1    0 003 
## 2    1 012 
## 3    0 102 
## 4    0 021D
## 5    0 021U
## 6    0 021C
## 7    0 111D
## 8    0 111U
## 9    0 030T
## 10   0 030C
## 11   0 201 
## 12   0 120D
## 13   0 120U
## 14   0 120C
## 15   0 210 
## 16   0 300
\end{verbatim}

\begin{Shaded}
\begin{Highlighting}[]
\NormalTok{two <-}\StringTok{ }\KeywordTok{graph_from_literal}\NormalTok{(C}\OperatorTok{-+}\NormalTok{B, B}\OperatorTok{-+}\NormalTok{A, A}\OperatorTok{-+}\NormalTok{C)}

\KeywordTok{data.frame}\NormalTok{(}\DataTypeTok{tri =} \KeywordTok{triad_census}\NormalTok{(two), }\DataTypeTok{name =}\NormalTok{ triad_names)}
\end{Highlighting}
\end{Shaded}

\begin{verbatim}
##    tri name
## 1    0 003 
## 2    0 012 
## 3    0 102 
## 4    0 021D
## 5    0 021U
## 6    0 021C
## 7    0 111D
## 8    0 111U
## 9    0 030T
## 10   1 030C
## 11   0 201 
## 12   0 120D
## 13   0 120U
## 14   0 120C
## 15   0 210 
## 16   0 300
\end{verbatim}

\begin{Shaded}
\begin{Highlighting}[]
\NormalTok{three <-}\StringTok{ }\KeywordTok{graph_from_literal}\NormalTok{(A}\OperatorTok{-+}\NormalTok{B, B}\OperatorTok{-+}\NormalTok{A, A}\OperatorTok{-+}\NormalTok{C)}

\KeywordTok{data.frame}\NormalTok{(}\DataTypeTok{tri =} \KeywordTok{triad_census}\NormalTok{(three), }\DataTypeTok{name =}\NormalTok{ triad_names)}
\end{Highlighting}
\end{Shaded}

\begin{verbatim}
##    tri name
## 1    0 003 
## 2    0 012 
## 3    0 102 
## 4    0 021D
## 5    0 021U
## 6    0 021C
## 7    0 111D
## 8    1 111U
## 9    0 030T
## 10   0 030C
## 11   0 201 
## 12   0 120D
## 13   0 120U
## 14   0 120C
## 15   0 210 
## 16   0 300
\end{verbatim}


\end{document}
